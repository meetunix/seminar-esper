\documentclass{beamer}

\usepackage[utf8]{inputenc}
\usepackage{ngerman}
\usepackage{amsmath}
%\usepackage[T1]{fontenc}
\usepackage[compress]{beamerthemeBoadilla}
\usecolortheme{rose}
\usepackage{graphicx}
\usepackage{colortbl}


\beamertemplatenavigationsymbolsempty

%Transparente Overlays:
%\setbeamercovered{transparent}

%Universitätssiegel:
%\pgfdeclareimage[height=0.7cm]{Siegel}{img/siegel.pdf}
%\logo{\pgfuseimage{Siegel}}

%%%%%%%%% FARBEN %%%%%%%%%%%

\definecolor{lightgray}{gray}{.92}
\definecolor{lightblue}{rgb}{0.8,0.85,1}
\definecolor{lightgreen}{rgb}{0.8,1,0.85}
\definecolor{lightred}{rgb}{1,0.75,0.7}
\definecolor{lightyellow}{rgb}{1,0.99,0.65}

%%Cellpadding
\setlength{\tabcolsep}{0.1cm}
\renewcommand{\arraystretch}{1.5}


\title{CEP mit Esper}
\subtitle{\Large Complex Event Processing am Beispiel von Esper}
\institute{}
\author{Martin Steinbach}
\date{14.01.2019}

\begin{document}


\part{Startfolie}
%Startfolie
\begin{frame}
\titlepage
\end{frame}

\begin{frame}{Agenda}
  
\end{frame}


\part{Praesentation}
%%%%%%%%%%%%%%%%%%%%%%%%%%%%%%%%%%%%%%%%%%%%%%%%%%%%%
\section{Grundlagen CEP}
\subsection{Was ist CEP?}
\begin{frame}{Testüberschrift}
pl
\end{frame}



%%%%%%%%%%%%%%%%%%% ESPER %%%%%%%%%%%%%%%%%%%%%%%%%%
\section{Esper}
\begin{frame}{Esper}
 \begin{exampleblock}{titel}
   Inhalt...
 \end{exampleblock}
\end{frame}
%%%%%%%%%%%%%%%%%%%%%%%%%%%%%%%%%%%%%%%%%%%%%%%%%%%%%
\part{Endfolie}
\setbeamercolor{background canvas}{bg=green!3}
\begin{frame}

\framebreak
\begin{center}
\Large \textcolor{red!60}{{Danke für die Aufmerksamkeit}}\\
\vspace{0.5 cm}
\end{center}
\begin{center}

\texttt{\scriptsize martin.steinbach@uni-rostock.de}\\

\end{center}



\end{frame}



\end{document}
